\documentclass[12pt,letterpaper]{article}
%\usepackage[left=12mm, top=0.5in, bottom=5in]{geometry}

\begin{document}
\title{University with no mobility barriers}
\author{No Way? No Way!}
%\date{ }
\maketitle

\begin{flushright}
\vspace*{\fill}
\textbf{Supervisor:} Agnieszka Wosiak\\
\textbf{Team Members:} Malwina Kruk\\ Jakub Karlo\\ Witold Derach\\ Kamil Glonek
\end{flushright}

\newpage

\renewcommand*\contentsname{Table of contents}
\tableofcontents

\newpage


\section{Executive summary}
This is Executicve summary.

\section{Introduction}
This is introduction section motherfucker.

\subsection{Team presentation}

\subsubsection{Members}

\subsubsection{Name and logo}

\subsection{Project's topic description}
\textbf{University with no mobility barriers?} \\\\
The past years have seen an influx of mobility apps with the aim of increasing and improving people's personal mobility. More people than ever use their smartphones to find information on how to move from A to B, which means of transport they should use rtc.. However, how does this work for people with a mobility impairment? Can mobility apps provide an alternative for people using a wheelchair or a walking frame, or even people with a stroller ar loads of luggage?\\\\
People with a mobility impairment need more information than usually provided by mobility apps. And because many apps do not provide this additional information, thay cannot use them.

\subsection{Problem definition}

\section{Conclusions}

\section{List of tables}

\section{Bibliography}

\section{References}

\end{document}