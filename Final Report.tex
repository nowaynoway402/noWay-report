\documentclass[12pt]{article}
\usepackage{graphicx}
\usepackage{float}


\begin{document}


% Title Page
\begin{titlepage}
\title{University with no mobility barriers}
\date {June 7, 2016}
\author{Team number 40}
\maketitle
\pagenumbering{empty}
\vfill
\raggedright
Supervisor: \\ Agnieszka Wosiak
\end{titlepage}

% Summary section

\section*{Executive Summary}
\addcontentsline{toc}{section}{\numberline{}Executive Summary}
The aim of this report is to
\pagenumbering{roman}
\cleardoublepage

% Table of contents
\tableofcontents
\thispagestyle{empty}

\newpage

% Introduction
\pagenumbering{arabic}
\setcounter{page}{2}
\section{Introduction}
In this section there is provided information about the team and its members, its supervisor, the roots of the team name and explanation of the logo.
\subsection{Team Members}
\paragraph{} Our team consists of four people. All of us are second year students of Information Technology at International Faculty of Engineering, which is part of Lodz Univeristy of Technology.
We are connected by the common goal, which is stated in the following part of the report. To achieve this goal, we were supervised during this semester by Ph.D Agnieszka Wosiak. The team members are:
\begin{itemize}
\item Malwina Kruk
\item Witold Derach
\item Kamil Glonek
\item Jakub Kar\l{}o
\end{itemize}

%Name
\subsection{Team Name}
\paragraph{} We decided to call our team "No Way? No Way!" The name reflects us and our attitude to the task. We strongly believe, that there is no problem which has no way to solve it. Furthermore, the name indicates that people with mobility impairments should not limit themselves. This issue will be explained in the problem statement section.

\newpage
%Logo

\begin{figure}[H]
\subsection{Team Logo}
\paragraph{}The logo presented below is a mixture of wheelchair and flame. Flame is considered by us as a symbol of power and strength. In addition, fire destroys any obstacle it encounters. In our opinion, people with mobility impairments should be like fire, they should not be afraid of any obstacles and overcome them. Due to that fact, we decided to replace the standard wheel with previously mentioned flame. The red colour, which dominates the logo, is commonly associated with energy, that is the reason for choosing it as primary colour.
\includegraphics[width=\textwidth]{"logo fire".png}
\caption{"No Way? No Way!" Team Logo}
\label{fig:logo}
\end{figure}

%Problem statement
\section{Problem statement}

\subsection{Project's topic description}
\textbf{University with no mobility barriers?} \\\\
\paragraph{}The past years have seen an influx of mobility apps with the aim of increasing and improving people's personal mobility. More people than ever use their smartphones to find information on how to move from A to B, which means of transport they should use etc.. However, how does this work for people with a mobility impairment? Can mobility apps provide an alternative for people using a wheelchair or a walking frame, or even people with a stroller ar loads of luggage?\\\\
People with a mobility impairment need more information than usually provided by mobility apps. And because many apps do not provide this additional information, thay cannot use them. That is why we decided to handle this situation and to find a solution that may be helpful to disabled people.
As a solution we decided to make an application, that will show the easiest root (root that has no curbs or trees in the middle of sidewalks) when traveling across our university. 



%From Witek
\subsection{Problem recognition}
\paragraph{}Our problem could be analyzed in many ways. This led us to a brainstorming after which we decided to use some of the easy-to-reach methods to find a solution. Our group decided that we need to get at least basic knowledge about every area. 
\subsubsection{Questionnaires}
\paragraph{}The first option was sending questionnaires to BON. Of course we met some difficulties. The first one was whether BON would be kind and sent it to the disabled people studying at our university. After several attempts the decided to help us with our problem and obliged themselves to send the queries to all students registered in BON. Unfortunately we had to formulates proper questions which will be easy to understand for everyone and not abusive. Many time spend on that part of the problem give us the expected result. Well prepared questionnaire with easy to understand questions. 
\subsubsection{Speaking with disabled people}
\paragraph{}Sending questionnaires was not enough for us, we were in need of getting some more information. Our group decided to divide ourselves by half and ask accidentally met people on the street with moving impairment about their challenges that they have to face every day when it comes about moving around the town. The answers were surprisingly decent and full of information about things that we did not take care of before. Before starting our own experiments we would not even think that segments like heavy or only wrongly build-in door could be a tough to overcome obstacle. Some respondents were even keen on showing us how it looks like in practice. We were moving with them around town to experience all the things they were talking about. 
\subsubsection{Making own experiments}
\paragraph{}As a group of future engineers and people curious about the world, we knew that one day we will do something unbelievable. Soon an option of renting a wheelchair arose. We knew that it will be unique. We made our minds as soon as possible to go and take it. For the next two days we have been driving on the wheelchairs through our university testing all the routes, pavements, entrances and facilities which have been set according to all disabled students at our university. The special elevators are a great and comfortable solution, but unfortunately not all of the buildings can have it. During using the wheelchair we have attached a gyroscope to see the changes in the height in level of area.   

\subsection{Possible solutions}
\paragraph{}When all our research has been completed. We decided to start thinking about a possible solution. As a creative group we had millions of ideas per minute but only small amount of them are worth mentioning. 

\subsubsection{Volunteering}
\paragraph{}The first idea that came to our minds was creating a small army of volunteers that will be ready to help students in need every time. Those students would receive special clothes prepared only for them. Those clothes will make the volunteers easy recognizable. After coming up to that point we saw a small bug that could not be fixed namely one day there may be no volunteers on our university. Each of volunteers has got an own life, other duties apart from studying. It would be rude to firstly give help and then leave people in need with nothing. Some students may forget how it was when they had to deal with problems on their own and what methods they used to overcome their problems which will make their life at university incredibly tough.

\subsubsection{Helpline}
\paragraph{}The next idea that came to our mind was a helpline. That solutions seems to be almost perfect. All students own a mobile phone and are able to call for help when such situation occurs. It is more effective when it comes to getting help, much faster, easily accessible and more effective. By using phone they may call some volunteers when such situation appears. On the other hand, the helpline may have similar problems with functioning as volunteering because, every student has got his own life. They may be having classes, driving a car, doing shopping so answering the call would be again impossible. That is why this option also has not been chosen by us. 

\subsubsection{Map of the university}
\paragraph{}After visiting the whole university and its buildings on wheelchairs we have pretty wide knowledge about the routes going through it. That is why the map if our university, came to our heads as the next solution. Unfortunately this problem must have been also rejected. If we decided to take a map as a form of a book it would be enormously heavy and insufficient to include all our maps. We could not imagine driving with one hand holding the map or changing the page. The other option in this place which is a huge map which is even worse than the previous one. The map lying on the knees of disabled person may fall on the ground, screw in the wheel and cause many more dangerous problems. Including all routes in one place would also cause many problems for the student with clarity of the map. 

\subsection{Final solution}
\paragraph{}After many hours of sitting and thinking about a perfect solution we all came up to one thing that may seem perfect. This solution is a mobile application so storing all the maps in one, single,  easily accessible place. As it was mentioned before all the students own a mobile phone, this device can be easily operated with one hand, it can lie on our knees, be in our pocket and it still can be used. Students may use it whenever they want without asking others for help. The only requirement is internet connection.  Student will just have to choose the initial and final destination and then the map with the route will be shown. The route shown is the best way for people moving on wheelchairs with the smallest amount of irregularity of terrain. The application is adapted for both the newest and the older version of mobile phones so if some students do not possess newest mobile device there is no worry because the application will be still working. 


\section{Application project and implementation}
%jakiś wstępik
\subsection{Preliminaries}
% Witek, Znajdowanie mapy, najłatwiejszej drogi, nasze założenia, połączenie z netem
\subsection{Tools}
\paragraph{} The tool we took advantage of to create and develop our application is Android Studio, the official Integrated Development Environment for Android apps. This IDE is based on IntelliJ IDEA by JetBrains. During our semester classes the part of our group worked in this environment, that is the main reason for our choice. In addition, the interface is very intuitive and some features can be added only by a few mouse clicks. It also provides the very useful mode, which is debugging on the connected device. It is significantly more convenient during checking gestures and touch responses.
\paragraph{} The decision of making an application for Android devices implicated the choice of programming language. The official language for Android development is Java. It is an object oriented language, which, according to Oracle, is the most used one in the world. What is more, it was derived from C++, which we have been learing since our first days at university. Consequently, we had some basics to get familiar with this language during the whole semester.
\paragraph{} The next important issue was the choice of minimum Software Development Kit (SDK) for our application. It was one of the most essential parts of the project as this determined amount of features and target devices. The higher the Application Programming Interface (API), the more features, but less target devices the application has. As our project is not taking advantage of the most modern technologies and solutions, we decided to concentrate on availability. As a result, the minimum SDK for our project is API 18: Android 4.3 (Jelly Bean). It means that our application will run on approximately 76.9\% devices that are active on the Google Play Store. 
On the other hand, if we chose API 23: Android 6.0 (Marshmallow), the application would run on only approximately 4.7\% devices. This data is provided by "Create New Project" dialog window in Android Studio, what can be noticed on the figures below.
\begin{figure}[H]
\includegraphics[width=\textwidth]{"API18".png}
\caption{API18 minimum SDK}
\label{fig:API18}
\end{figure}
\begin{figure}[H]
\includegraphics[width=\textwidth]{"API23".png}
\caption{API23 minimum SDK}
\label{fig:API23}
\end{figure}

\subsection{Application components}
% screeny z apki i opis jak działajo
\subsubsection{Main menu}
%Malwina
\subsubsection{Map}
\paragraph{} This is the main component of the application. This part consists of the map view and user interface above with two select boxes and "Find Path" button. The initial state of that module is presented in the figure below.

\begin{figure}[H]
\centerline{\includegraphics[width=250px]{"Map initial state".png}}
\caption{Map initial state}
\label{fig:MapInitial}
\end{figure}

\paragraph{} After choosing the points, the application displays path on a map alongside with the distance and approximate time needed to walk through the path. The exemplary application output after this operation is shown below.

\begin{figure}[H]
\centerline{\includegraphics[width=250px]{"Map with drawn path".png}}
\caption{Map with a specific path}
\label{fig:MapPath}
\end{figure}
 
\subsubsection{Help and About}
%Witek

\subsection{Implementation}
% Malwina aplikacja składa się z activities i xml itp
\subsubsection{Activities}
% Malwina, tu wrzucamy wszystkie 5 activities i je opisujemy
\subsubsection{Map algorithm}
% Kamil Kuba
\subsubsection{Layout}
% Malwina

%From Malwina
\section{Conclusions}
\paragraph{}
The problem defined at the beginning of this report sounded ``University with no mobility barriers''.
In the consequence we developed the solution in the shape of the mobile application.

However, the main goal of the project, apart from presenting tangible product which answers the needs of the target group, was seeking for ways to do this and establishing reliable relationship within our team.
The conclusions and our resultant estimation of closing outcome are contained in the sections below.

\subsection{Conclusions concerning solution}
\paragraph{}
The problem defined challenged us to deal with the daily difficulties of disabled people.
After broad research and strict selection of such, our team decided to create a specific tool.
The mobile application cannot eliminate these handicaps, although, it may provide ways to overcome them.

\paragraph{}
The application, as presumed, supply a user with an interactive map connected to the Internet.
With this component one can:
\begin{itemize}
	\item locate any building on both of the University campuses,
	\item find the most suitable entry for their needs (program marks entries with driveway for the disabled),
	\item choose the easiest path between two points.
\end{itemize}
The aspects mentioned were included into assumed results.
The program facilitates movements within and around campuses of Lodz University of Technology.
Algorithm contained in application structure calculates specific routes enabling traveling.

\paragraph{}
Our purpose to prepare tracks available for the disabled people was achieved.
The application creates paths on the basis of data marking places that may produce trouble during passage.
Proper section of track will not cross any of these points and will lead a user the path of the smallest fluctuation in height.
Light and legible layout eases manipulating map component and reading the instructions from the screen.
Large buttons allow comfortable operating of the whole application.
We also supplied users with special content helping them to learn this new tool.

Section ``Help'' shows anyone how to move inside the application and how to initiate the process of path computing.
There is also a phone number to the Diabled People Office, helping handicaped students with any problems.
Under any circumstances a student in need can contact this centre.
In any other matters and issues involving application service user will find contact information to the authors.

This application was prepared with great diligence.
It fulfils the assumed guidelines and creates a useful instrument in the aspect of student's life.

\subsection{Conclusions concerning team building}
\paragraph{}
The main purpose of this project and side effect of creating the solution was developing soft skills.
We were to learn and adopt abilities to cooperate effectively within a team.
Working on the product resulted in better understanding basic mechanisms standing behind establishing good relationships.

\paragraph{}
We come to learn:
\begin{itemize}
	\item active listening,
	\item ability to resolve conflicts,
	\item ability to reach compromise,
	\item fair distribution of work,
	\item supporting the team-mates.
\end{itemize}

\paragraph{}
Different opinions, especially when analysing the problem and possible solutions, brought us to the point where negotiations and peaceful discussion was needed.
By means of such situations we worked out schemas to properly achieve consensus.
Apart from absorbing the knowledge we managed to find our own ways helping us to cooperate.
We agreed over the fact that our personal and specially adapted rules, and by this created relationship, is far better than strictly imposed structure.
With a great labour and a lot of willpower we decided to rely collaboration on our friendship.
Trying to artificially create working atmosphere with stiff system of relations would only damage the greatest advantage of our team which is personal, emotional bond.

Equality, peace and trail how to create a good team made us an successful group.
Our skills and hard work were only the effect of effort done to build a team.

\section{Recommendations}
\paragraph{}
 For the future of mobile application ``No way? No way!'' we want to present our recommendations and possible ways of development.
These suggestions when implemented will not affect any functionality of present program.
They will only improve activity and expand potential features.

\begin{itemize}
	\item There can be provided another functionality allowing to choose between the effortless path and the fast one.
	The latter will be obtained on the basis of distance from destination.
	In other case, this route may be calculated using time of travelling (crossroads or areas of increased traffic might extend the time of passing).
	This solution could be convenient for people prepared for though roads or in a hurry.
	
	\item Prospective property is adjusting application for any device.
	People trying to plan their trip will find it opportune to check route earlier on their PCs, MACs, etc.
	
	\item Any linkage between the program and other applications adding new features would be beneficial.
	Special extensions with the map component attached to other mobile applications will only be advantageous for students and producers.
	With this thing anyone using any program utilising maps would be able to apply our solution.
\end{itemize}

Development is a significant aspect of every product.
Users expect it to move with the evolution of technology.
To remain useful product must apply new features and deal with possible mistakes or problems.

Due to consistency and transparency of the application future development is possible.
Owing to Internet connection sending patches, extension packs will be simple.
The only step before is creating new functionalities.

\newpage
\listoffigures

\newpage
\begin{thebibliography}{9}
\bibitem{androidUserGuide}
Android Studio User Guide
\textit {https://developer.android.com/studio/intro/index.html}

\end{thebibliography}

\end{document}

